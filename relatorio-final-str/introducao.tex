Na medida em que o uso de sistemas computacionais aumenta na sociedade atual, aplicações com exigências de tempo real tornam-se cada vez mais comuns. Essas aplicações variam muito em relação à complexidade e às necessidades de garantia no entendimento de restrições temporais. Existem sistemas de tempo real simples, como os controladores inteligentes embutidos em utilidades domésticas, tais como lavadora de roupa e videocassetes e existem sistemas de complexidade maior, como os sistemas militares de defesa, sistemas de controle de plantas industriais (químicas e nucleares) e controle de tráfego aéreo e ferroviário.

Em aplicações de tempo real, também é importante ressaltar que existem sistemas que apresentam restrições de tempo mais rigorosas do que outras, por exem- plo, os sistemas responsáveis pelo monitoramento de pacientes em hospitais, sistemas embarcados em robôs e veículos (de automóveis até aviões) e há também sistemas que não apresentam restrições tão rigorosas, por exemplo, videogames, teleconferências através da internet e aplicações de multimídia. Todas essas aplicações que apresentam a característica adicional de estarem sujeitas a restrições temporais são agrupadas no que é usualmente identificado como Sistema de Tempo Real.

Diante dessa premissa, com o intuito de aplicar os conhecimentos adquiridos na disciplina de Sistemas de Tempo Real desenvolvemos neste projeto um robô que possui tarefas a serem executadas em tempo real.  Utilizamos a plataforma Arduino, diversos sensores e uma biblioteca para auxiliar no escalonamento de tarefas. Neste documento são detalhadas todas as etapas do desenvolvimento do robô. O documento foi organizado da seguinte forma: A seção 2 explica como o robô foi construído, a seção 3 apresenta as tarefas do robô e mostra como elas foram implementadas, a seção 4 expõe os resultados obtidos e por fim, a seção 5 apresenta a conclusão do trabalho. 